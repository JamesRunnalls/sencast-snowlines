%% Generated by Sphinx.
\def\sphinxdocclass{report}
\documentclass[letterpaper,10pt,english]{sphinxmanual}
\ifdefined\pdfpxdimen
   \let\sphinxpxdimen\pdfpxdimen\else\newdimen\sphinxpxdimen
\fi \sphinxpxdimen=.75bp\relax

\PassOptionsToPackage{warn}{textcomp}
\usepackage[utf8]{inputenc}
\ifdefined\DeclareUnicodeCharacter
% support both utf8 and utf8x syntaxes
  \ifdefined\DeclareUnicodeCharacterAsOptional
    \def\sphinxDUC#1{\DeclareUnicodeCharacter{"#1}}
  \else
    \let\sphinxDUC\DeclareUnicodeCharacter
  \fi
  \sphinxDUC{00A0}{\nobreakspace}
  \sphinxDUC{2500}{\sphinxunichar{2500}}
  \sphinxDUC{2502}{\sphinxunichar{2502}}
  \sphinxDUC{2514}{\sphinxunichar{2514}}
  \sphinxDUC{251C}{\sphinxunichar{251C}}
  \sphinxDUC{2572}{\textbackslash}
\fi
\usepackage{cmap}
\usepackage[T1]{fontenc}
\usepackage{amsmath,amssymb,amstext}
\usepackage{babel}



\usepackage{times}
\expandafter\ifx\csname T@LGR\endcsname\relax
\else
% LGR was declared as font encoding
  \substitutefont{LGR}{\rmdefault}{cmr}
  \substitutefont{LGR}{\sfdefault}{cmss}
  \substitutefont{LGR}{\ttdefault}{cmtt}
\fi
\expandafter\ifx\csname T@X2\endcsname\relax
  \expandafter\ifx\csname T@T2A\endcsname\relax
  \else
  % T2A was declared as font encoding
    \substitutefont{T2A}{\rmdefault}{cmr}
    \substitutefont{T2A}{\sfdefault}{cmss}
    \substitutefont{T2A}{\ttdefault}{cmtt}
  \fi
\else
% X2 was declared as font encoding
  \substitutefont{X2}{\rmdefault}{cmr}
  \substitutefont{X2}{\sfdefault}{cmss}
  \substitutefont{X2}{\ttdefault}{cmtt}
\fi


\usepackage[Bjarne]{fncychap}
\usepackage{sphinx}

\fvset{fontsize=\small}
\usepackage{geometry}


% Include hyperref last.
\usepackage{hyperref}
% Fix anchor placement for figures with captions.
\usepackage{hypcap}% it must be loaded after hyperref.
% Set up styles of URL: it should be placed after hyperref.
\urlstyle{same}

\addto\captionsenglish{\renewcommand{\contentsname}{Installation}}

\usepackage{sphinxmessages}
\setcounter{tocdepth}{1}



\title{Sentinel\sphinxhyphen{}Hindcast}
\date{Jun 29, 2020}
\release{}
\author{Daniel Odermatt, James Runnalls, Rolf Scheuner}
\newcommand{\sphinxlogo}{\vbox{}}
\renewcommand{\releasename}{}
\makeindex
\begin{document}

\pagestyle{empty}
\sphinxmaketitle
\pagestyle{plain}
\sphinxtableofcontents
\pagestyle{normal}
\phantomsection\label{\detokenize{index::doc}}


\noindent{\sphinxincludegraphics[width=120\sphinxpxdimen]{{logo}.png}\hspace*{\fill}}

Sentinel Hindcast is a python toolbox that forms a framework around existing packages for processing
Sentinel 2 \& Sentinel 3 satellite images in order facilitates processing pipelines for deriving water
quality parameters such as Chlorophyll A, Turbidity, etc.

It is developed and maintained by the \sphinxhref{https://www.eawag.ch/en/department/surf/main-focus/remote-sensing/}{SURF Remote Sensing group at Eawag}.

\begin{sphinxadmonition}{warning}{Warning:}
Sentinel hindcast is under active development. The project team are working towards
the release of a stable v1.0, however for the moment this project remains pre\sphinxhyphen{}v1.0.
\end{sphinxadmonition}


\chapter{Installation}
\label{\detokenize{index:installation}}
To install sentinel\sphinxhyphen{}hindcast, run:

\begin{sphinxVerbatim}[commandchars=\\\{\}]
\PYG{n}{git} \PYG{n}{clone} \PYG{n}{git}\PYG{n+nd}{@renkulab}\PYG{o}{.}\PYG{n}{io}\PYG{p}{:}\PYG{n}{odermatt}\PYG{o}{/}\PYG{n}{sentinel}\PYG{o}{\PYGZhy{}}\PYG{n}{hindcast}\PYG{o}{.}\PYG{n}{git}
\PYG{n}{pip} \PYG{n}{install} \PYG{o}{\PYGZhy{}}\PYG{n}{r} \PYG{n}{requirements}\PYG{o}{.}\PYG{n}{txt}
\end{sphinxVerbatim}

Many of the Sentinel\sphinxhyphen{}Hindcast processors reply on \sphinxhref{http://step.esa.int/main/toolboxes/snap/}{SNAP} , the SeNtinel Application Platform
project, funded by the \sphinxhref{http://www.esa.int/}{European Space Agency} (ESA) or other 3rd party packages. In order to have
access to all of Sentinel Hindcast’s processors follow the installation instructions below in order to
correctly configure your environment.

This process will require registering accounts with data providers.
\begin{itemize}
\item {} 
{\hyperref[\detokenize{ubuntu18_install:ubuntu18install}]{\sphinxcrossref{\DUrole{std,std-ref}{Ubuntu 18}}}}

\item {} 
{\hyperref[\detokenize{centos8_install:centos8install}]{\sphinxcrossref{\DUrole{std,std-ref}{CentOS 8}}}}

\item {} 
{\hyperref[\detokenize{windows10_install:windows10install}]{\sphinxcrossref{\DUrole{std,std-ref}{Windows 10}}}}

\end{itemize}

For issues with installation, please contact \sphinxhref{https://www.eawag.ch/de/ueberuns/portraet/organisation/mitarbeitende/profile/daniel-odermatt/show/}{Daniel
Odermatt}.


\chapter{Getting Started}
\label{\detokenize{index:getting-started}}

\section{Environment File}
\label{\detokenize{index:environment-file}}
Environment files use the INI format and contain the configuration of
the machine on which Sentinel Hindcast runs. Refer to {\hyperref[\detokenize{environment_config:environments}]{\sphinxcrossref{\DUrole{std,std-ref}{Environment File}}}}
for details on how to set up your own environment file.

You should create your own environment file for every machine you
install Sentinel Hindcast on.


\section{Parameter File}
\label{\detokenize{index:parameter-file}}
Parameter files use the INI format and contain the parameters for one
execution of Sentinel Hindcast. Refer to {\hyperref[\detokenize{parameters_config:parameters}]{\sphinxcrossref{\DUrole{std,std-ref}{Parameter File}}}}
for details on how to set up your own parameter file.


\section{Perimeter Definition}
\label{\detokenize{index:perimeter-definition}}
Perimeter definitions define a geographic area to be processed by
Sentinel Hndacast. They are stored as polygons in \sphinxhref{https://en.wikipedia.org/wiki/Well-known\_text\_representation\_of\_geometry}{WKT} files, which
are referenced from the parameter files. Some example perimeters are stored
in the wkt folder.


\section{Data Processing}
\label{\detokenize{index:data-processing}}
Data is preprocessed by a build\sphinxhyphen{}in preprocessor which performs
resampling, subsetting, idepix and reproject operations on the input
products. Several processors then process the data and save the results
to disk.

Sentinel Hindcast offers to interfaces to process data.
\begin{itemize}
\item {} 
The file\sphinxhyphen{}based interface takes a parameter file and an optional
environment file as input. It reads the file contents and calls the
object based interface with the read configurations.

\item {} 
The object\sphinxhyphen{}based interface directly takes an environment and a
parameters object as well as a path for the L1 (input) products and a
path for the L2 (output) products.

\end{itemize}


\section{Adapters}
\label{\detokenize{index:adapters}}
Adapters can receive the output of processors and for example send it to
another service.


\subsection{Ubuntu 18}
\label{\detokenize{ubuntu18_install:ubuntu-18}}\label{\detokenize{ubuntu18_install:ubuntu18install}}\label{\detokenize{ubuntu18_install::doc}}
2.) OpenJdk: \sphinxurl{https://dzone.com/articles/installing-openjdk-11-on-ubuntu-1804-for-real} (if not installed already):
\begin{quote}
\begin{description}
\item[{In shell do following:}] \leavevmode\begin{description}
\item[{\$ sudo apt\sphinxhyphen{}get install default\sphinxhyphen{}jdk}] \leavevmode
\textgreater{} y

\item[{\$ java \sphinxhyphen{}version}] \leavevmode
Needs to be version 11

\end{description}

\end{description}
\end{quote}

3.) Maven: \sphinxurl{https://www.javahelps.com/2017/10/install-apache-maven-on-linux.html}
\begin{quote}
\begin{description}
\item[{In shell do following:}] \leavevmode
\$ mkdir \sphinxhyphen{}p \textasciitilde{}/Downloads
\$ curl \sphinxurl{http://mirror.easyname.ch/apache/maven/maven-3/3.6.3/binaries/apache-maven-3.6.3-bin.tar.gz} \sphinxhyphen{}o \textasciitilde{}/Downloads/apache\sphinxhyphen{}maven\sphinxhyphen{}3.6.3\sphinxhyphen{}bin.tar.gz
\$ sudo tar \sphinxhyphen{}xvzf \textasciitilde{}/Downloads/apache\sphinxhyphen{}maven\sphinxhyphen{}3.6.3\sphinxhyphen{}bin.tar.gz \textendash{}directory /home/jamesrunnalls
\$ sudo su \sphinxhyphen{}c ‘echo “M2\_HOME=/home/jamesrunnalls/apache\sphinxhyphen{}maven\sphinxhyphen{}3.6.3/” \textgreater{}\textgreater{} /etc/environment’
\$ sudo update\sphinxhyphen{}alternatives \textendash{}install “/usr/bin/mvn” “mvn” “/home/jamesrunnalls/apache\sphinxhyphen{}maven\sphinxhyphen{}3.6.3/bin/mvn” 0
\$ sudo update\sphinxhyphen{}alternatives \textendash{}set mvn /home/jamesrunnalls/apache\sphinxhyphen{}maven\sphinxhyphen{}3.6.3/bin/mvn
\$ mvn \sphinxhyphen{}version
\$ sudo reboot

\end{description}
\end{quote}

4.) Anaconda: \sphinxurl{https://problemsolvingwithpython.com/01-Orientation/01.05-Installing-Anaconda-on-Linux/}
\begin{quote}
\begin{description}
\item[{In shell do following:}] \leavevmode
\$ curl \sphinxurl{https://repo.anaconda.com/archive/Anaconda3-2020.02-Linux-x86\_64.sh} \sphinxhyphen{}o \textasciitilde{}/Downloads/Anaconda3\sphinxhyphen{}2020.02\sphinxhyphen{}Linux\sphinxhyphen{}x86\_64.sh
\$ sudo chmod 777 /home/jamesrunnalls
\$ bash \textasciitilde{}/Downloads/Anaconda3\sphinxhyphen{}2020.02\sphinxhyphen{}Linux\sphinxhyphen{}x86\_64.sh

\begin{sphinxVerbatim}[commandchars=\\\{\}]
\PYG{g+gp}{\PYGZgt{}\PYGZgt{}\PYGZgt{} }\PYG{p}{[}\PYG{n}{Enter}\PYG{p}{]}
\PYG{g+go}{[s]}
\PYG{g+gp}{\PYGZgt{}\PYGZgt{}\PYGZgt{} }\PYG{n}{yes}
\PYG{g+gp}{\PYGZgt{}\PYGZgt{}\PYGZgt{} }\PYG{o}{/}\PYG{n}{home}\PYG{o}{/}\PYG{n}{jamesrunnalls}\PYG{o}{/}\PYG{n}{anaconda3}
\PYG{g+gp}{\PYGZgt{}\PYGZgt{}\PYGZgt{} }\PYG{n}{yes}
\end{sphinxVerbatim}

\$ sudo chmod 755 /home/jamesrunnalls
\$ sudo reboot

\end{description}
\end{quote}
\begin{enumerate}
\sphinxsetlistlabels{\arabic}{enumi}{enumii}{}{.}%
\setcounter{enumi}{4}
\item {} 
Anaconda: create sentinel\sphinxhyphen{}hindcast\sphinxhyphen{}37 environment
\begin{quote}
\begin{description}
\item[{In shell do following:}] \leavevmode
\$ conda config \textendash{}add channels conda\sphinxhyphen{}forge
\$ conda create \textendash{}name sentinel\sphinxhyphen{}hindcast\sphinxhyphen{}37 python=3.7 gdal cartopy netcdf4 cython pkgconfig statsmodels matplotlib haversine
\begin{quote}

\textgreater{} y
\end{quote}

\end{description}
\end{quote}

\end{enumerate}

6.) SNAP: \sphinxurl{http://step.esa.int/main/download/}
\begin{quote}

Uninstall all old versions of SNAP and remove associated data
\begin{description}
\item[{In shell do following:}] \leavevmode
\$ curl \sphinxurl{http://step.esa.int/downloads/7.0/installers/esa-snap\_all\_unix\_7\_0.sh} \sphinxhyphen{}o \textasciitilde{}/Downloads/esa\sphinxhyphen{}snap\_all\_unix\_7\_0.sh
\$ sudo chmod 777 /home/jamesrunnalls
\$ bash \textasciitilde{}/Downloads/esa\sphinxhyphen{}snap\_all\_unix\_7\_0.sh
\begin{quote}

{[}o, Enter{]}
{[}1, Enter{]}
{[}Enter{]}
{[}Enter{]}
{[}n, Enter{]}
{[}n, Enter{]}
{[}n, Enter{]}
\end{quote}

\$ sudo chmod 755 /home/jamesrunnalls
\$ /home/jamesrunnalls/snap/bin/snap \textendash{}nosplash \textendash{}nogui \textendash{}modules \textendash{}update\sphinxhyphen{}all
\$ /home/jamesrunnalls/snap/bin/snap \textendash{}nosplash \textendash{}nogui \textendash{}modules \textendash{}install org.esa.snap.idepix.core org.esa.snap.idepix.probav org.esa.snap.idepix.modis org.esa.snap.idepix.spotvgt org.esa.snap.idepix.landsat8 org.esa.snap.idepix.viirs org.esa.snap.idepix.olci org.esa.snap.idepix.seawifs org.esa.snap.idepix.meris org.esa.snap.idepix.s2msi
\$ echo “\#SNAP configuration ‘s3tbx’” \textgreater{}\textgreater{} \textasciitilde{}/.snap/etc/s3tbx.properties
\$ echo “\#Fri Mar 27 12:55:00 CET 2020” \textgreater{}\textgreater{} \textasciitilde{}/.snap/etc/s3tbx.properties
\$ echo “s3tbx.reader.olci.pixelGeoCoding=true” \textgreater{}\textgreater{} \textasciitilde{}/.snap/etc/s3tbx.properties
\$ echo “s3tbx.reader.meris.pixelGeoCoding=true” \textgreater{}\textgreater{} \textasciitilde{}/.snap/etc/s3tbx.properties
\$ echo “s3tbx.reader.slstrl1b.pixelGeoCodings=true” \textgreater{}\textgreater{} \textasciitilde{}/.snap/etc/s3tbx.properties

\end{description}

Note: there are many strange error messages, but it seems to work in the end when updating and installing plugins
\begin{description}
\item[{To remove warning “WARNING: org.esa.snap.dataio.netcdf.util.MetadataUtils: Missing configuration property ‘snap.dataio.netcdf.metadataElementLimit’. Using default (100).”:}] \leavevmode
\$ echo “” \textgreater{}\textgreater{} /home/jamesrunnalls/snap/etc/snap.properties
\$ echo “\# NetCDF options” \textgreater{}\textgreater{} /home/jamesrunnalls/snap/etc/snap.properties
\$ echo “snap.dataio.netcdf.metadataElementLimit=10000” \textgreater{}\textgreater{} /home/jamesrunnalls/snap/etc/snap.properties

\item[{To remove warning “SEVERE: org.esa.s2tbx.dataio.gdal.activator.GDALDistributionInstaller: The environment variable LD\_LIBRARY\_PATH is not set. It must contain the current folder ‘.’.”}] \leavevmode
\$ sudo su \sphinxhyphen{}c ‘echo “LD\_LIBRARY\_PATH=.” \textgreater{}\textgreater{} /etc/environment’

\end{description}
\end{quote}

7.) Python \sphinxhyphen{} jpy: \sphinxurl{https://github.com/bcdev/jpy/blob/master/README.md}
\begin{quote}
\begin{description}
\item[{In shell do following:}] \leavevmode
\$ sudo apt\sphinxhyphen{}get install python\sphinxhyphen{}setuptools
\$ cd /home/jamesrunnalls/anaconda3/envs/sentinel\sphinxhyphen{}hindcast\sphinxhyphen{}37/lib/python3.7/site\sphinxhyphen{}packages
\$ git clone \sphinxurl{https://github.com/bcdev/jpy.git}
\$ cd jpy
\$ conda activate sentinel\sphinxhyphen{}hindcast\sphinxhyphen{}37
\$ conda install \sphinxhyphen{}c conda\sphinxhyphen{}forge wheel
\$ python get\sphinxhyphen{}pip.py
\$ python setup.py build maven bdist\_wheel

\end{description}
\end{quote}

8.) Python \sphinxhyphen{} snappy: \sphinxurl{https://github.com/senbox-org/snap-engine/blob/master/snap-python/src/main/resources/README.md}
\begin{quote}
\begin{description}
\item[{In shell do following:}] \leavevmode
a\$ sudo ln \sphinxhyphen{}s ../../lib64/libnsl.so.2 /usr/lib64/libnsl.so
a\$ sudo ln \sphinxhyphen{}s ../../lib64/libnsl.so.2.0.0 /usr/lib64/libnsl.so.1
\$ mkdir \sphinxhyphen{}p \textasciitilde{}/.snap/snap\sphinxhyphen{}python/snappy
\$ cp /home/jamesrunnalls/anaconda3/envs/sentinel\sphinxhyphen{}hindcast\sphinxhyphen{}37/lib/python3.7/site\sphinxhyphen{}packages/jpy/dist/{\color{red}\bfseries{}*}.whl \textasciitilde{}/.snap/snap\sphinxhyphen{}python/snappy
\$ bash /home/jamesrunnalls/snap/bin/snappy\sphinxhyphen{}conf /home/jamesrunnalls/anaconda3/envs/sentinel\sphinxhyphen{}hindcast\sphinxhyphen{}37/bin/python \textasciitilde{}/.snap/snap\sphinxhyphen{}python
\$ conda activate sentinel\sphinxhyphen{}hindcast\sphinxhyphen{}37
\$ python \textasciitilde{}/.snap/snap\sphinxhyphen{}python/snappy/setup.py install \textendash{}user
\$ cp \sphinxhyphen{}avr \textasciitilde{}/.snap/snap\sphinxhyphen{}python/build/lib/snappy /home/jamesrunnalls/anaconda3/envs/sentinel\sphinxhyphen{}hindcast\sphinxhyphen{}37/lib/python3.7/site\sphinxhyphen{}packages/snappy
\$ cp \sphinxhyphen{}avr \textasciitilde{}/.snap/snap\sphinxhyphen{}python/snappy/tests /home/jamesrunnalls/anaconda3/envs/sentinel\sphinxhyphen{}hindcast\sphinxhyphen{}37/lib/python3.7/site\sphinxhyphen{}packages/snappy/tests
\$ cd /home/jamesrunnalls/anaconda3/envs/sentinel\sphinxhyphen{}hindcast\sphinxhyphen{}37/lib/python3.7/site\sphinxhyphen{}packages/snappy/tests
\$ curl \sphinxurl{https://raw.githubusercontent.com/bcdev/eo-child-gen/master/child-gen-N1/src/test/resources/com/bc/childgen/MER\_RR\_\_1P.N1} \sphinxhyphen{}o MER\_RR\_\_1P.N1
\$ python test\_snappy\_mem.py
\$ python test\_snappy\_perf.py
\$ python test\_snappy\_product.py

\end{description}
\end{quote}

9.) Python \sphinxhyphen{} polymer: \sphinxurl{https://forum.hygeos.com/viewforum.php?f=5}
\begin{quote}
\begin{description}
\item[{From a computer in the eawag network, copy the polymer zip file to the linux server:}] \leavevmode
\textgreater{} scp \sphinxhyphen{}i .sshcloudferro.key \textbackslash{}eawagAbteilungs\sphinxhyphen{}ProjekteSurfsurf\sphinxhyphen{}DDRSSoftwarePolymerpolymer\sphinxhyphen{}v4.13.tar.gz \sphinxhref{mailto:eouser@45.130.29.115}{eouser@45.130.29.115}:/home/eouser/Downloads

\item[{In shell do following:}] \leavevmode
a\$ sudo chmod 777 /home/jamesrunnalls
\$ tar \sphinxhyphen{}xvzf /home/jamesrunnalls/Downloads/polymer\sphinxhyphen{}v4.13.tar.gz \textendash{}directory /home/jamesrunnalls
a\$ sudo chmod 755 /home/jamesrunnalls
\$ cd /home/jamesrunnalls/polymer\sphinxhyphen{}v4.13
\$ conda activate sentinel\sphinxhyphen{}hindcast\sphinxhyphen{}37
\$ conda install \sphinxhyphen{}c conda\sphinxhyphen{}forge python=3 cython numpy pyhdf scipy netcdf4 pandas avalentino::pyepr glymur pyproj lxml gdal pygrib bioconda::ecmwfapi cdsapi xarray urllib3 pytest
\$ sudo apt\sphinxhyphen{}get install wget
\$ make all
\$ cp \sphinxhyphen{}avr /home/jamesrunnalls/polymer\sphinxhyphen{}v4.13/polymer /home/jamesrunnalls/anaconda3/envs/sentinel\sphinxhyphen{}hindcast\sphinxhyphen{}37/lib/python3.7/site\sphinxhyphen{}packages/polymer
\$ cp \sphinxhyphen{}avr /home/jamesrunnalls/polymer\sphinxhyphen{}v4.13/auxdata /home/jamesrunnalls/anaconda3/envs/sentinel\sphinxhyphen{}hindcast\sphinxhyphen{}37/lib/python3.7/site\sphinxhyphen{}packages/auxdata

\end{description}
\end{quote}

10.) sentinel\sphinxhyphen{}hindcast: \sphinxurl{https://renkulab.io/gitlab/odermatt/sentinel-hindcast}
\begin{quote}
\begin{description}
\item[{In shell do following:}] \leavevmode
\$ cd /prj
\$ sudo chmod 777 /prj
\$ mkdir /prj/DIAS
\$ git clone \sphinxurl{https://renkulab.io/gitlab/odermatt/sentinel-hindcast.git}
\$ sudo chmod 755 /prj
\$ cd sentinel\sphinxhyphen{}hindcast
\$ git checkout \textless{}branchname\textgreater{} (if not master)

\end{description}
\end{quote}

11.) CDS API: \sphinxurl{https://cds.climate.copernicus.eu/api-how-to}
\begin{quote}

Have a Copernicus Climate account ready, otherwise create one: \sphinxurl{https://cds.climate.copernicus.eu/}
\begin{description}
\item[{In shell do following:}] \leavevmode
\$ echo “url: \sphinxurl{https://cds.climate.copernicus.eu/api/v2}” \textgreater{} \textasciitilde{}/.cdsapirc
\$ echo key: {[}uid{]}:{[}api\sphinxhyphen{}key{]} \textgreater{}\textgreater{} \textasciitilde{}/.cdsapirc (Note: replace {[}uid{]} and {[}api\sphinxhyphen{}key{]} by your actual credentials, see \sphinxurl{https://cds.climate.copernicus.eu/api-how-to} )
\$ chmod 600 \textasciitilde{}/.cdsapirc

\end{description}
\end{quote}

12.) Cronjob for datalakes: \sphinxurl{https://linux4one.com/how-to-set-up-cron-job-on-centos-8/}
\begin{quote}
\begin{description}
\item[{In shell do following:}] \leavevmode
\$ mkdir \sphinxhyphen{}p /prj/datalakes/log
\$ curl \sphinxurl{https://renkulab.io/gitlab/odermatt/sentinel-hindcast/raw/snap7compatibility/parameters/datalakes\_sui\_S3.ini?inline=false} \sphinxhyphen{}o /prj/datalakes/datalakes\_sui\_S3.ini
\$ chmod 755 /prj/sentinel\sphinxhyphen{}hindcast/scripts/datalakes.sh
\$ crontab \sphinxhyphen{}l | \{ cat; echo “0 20 * * * nohup /prj/sentinel\sphinxhyphen{}hindcast/scripts/datalakes.sh \&”; \} | crontab \sphinxhyphen{}

\end{description}
\end{quote}

13.) (not done yet) NASA Earthdata API: \sphinxurl{https://wiki.earthdata.nasa.gov/display/EL/How+To+Access+Data+With+cURL+And+Wget}
\begin{quote}

Have a NASA Earthdata account ready, otherwise create one: \sphinxurl{https://urs.earthdata.nasa.gov/}
\begin{description}
\item[{In shell do following:}] \leavevmode
\$ touch \textasciitilde{}/.netrc
\$ echo “machine urs.earthdata.nasa.gov login \textless{}earthdata user\textgreater{} password \textless{}earthdata password\textgreater{}” \textgreater{} \textasciitilde{}/.netrc
\$ chmod 0600 \textasciitilde{}/.netrc
\$ touch \textasciitilde{}/.urs\_cookies

\end{description}
\end{quote}


\subsection{CentOS 8}
\label{\detokenize{centos8_install:centos-8}}\label{\detokenize{centos8_install:centos8install}}\label{\detokenize{centos8_install::doc}}
2.) OpenJdk (if not installed already):
\begin{quote}
\begin{description}
\item[{In shell do following:}] \leavevmode\begin{description}
\item[{\$ sudo yum install java\sphinxhyphen{}11\sphinxhyphen{}openjdk\sphinxhyphen{}devel}] \leavevmode
\textgreater{} y

\end{description}

\$ sudo su \sphinxhyphen{}c ‘echo “JAVA\_HOME=/usr/lib/jvm/java\sphinxhyphen{}11\sphinxhyphen{}openjdk/” \textgreater{}\textgreater{} /etc/environment’
\$ java \sphinxhyphen{}version
\$ sudo reboot

\end{description}
\end{quote}

3.) Maven: \sphinxurl{https://www.javahelps.com/2017/10/install-apache-maven-on-linux.html}
\begin{quote}
\begin{description}
\item[{In shell do following:}] \leavevmode
\$ mkdir \sphinxhyphen{}p \textasciitilde{}/Downloads
\$ curl \sphinxurl{http://mirror.easyname.ch/apache/maven/maven-3/3.6.3/binaries/apache-maven-3.6.3-bin.tar.gz} \sphinxhyphen{}o \textasciitilde{}/Downloads/apache\sphinxhyphen{}maven\sphinxhyphen{}3.6.3\sphinxhyphen{}bin.tar.gz
\$ sudo tar \sphinxhyphen{}xvzf \textasciitilde{}/Downloads/apache\sphinxhyphen{}maven\sphinxhyphen{}3.6.3\sphinxhyphen{}bin.tar.gz \textendash{}directory /opt
\$ sudo su \sphinxhyphen{}c ‘echo “M2\_HOME=/opt/apache\sphinxhyphen{}maven\sphinxhyphen{}3.6.3/” \textgreater{}\textgreater{} /etc/environment’
\$ sudo update\sphinxhyphen{}alternatives \textendash{}install “/usr/bin/mvn” “mvn” “/opt/apache\sphinxhyphen{}maven\sphinxhyphen{}3.6.3/bin/mvn” 0
\$ sudo update\sphinxhyphen{}alternatives \textendash{}set mvn /opt/apache\sphinxhyphen{}maven\sphinxhyphen{}3.6.3/bin/mvn
\$ mvn \sphinxhyphen{}version
\$ sudo reboot

\end{description}
\end{quote}

4.) Anaconda: \sphinxurl{https://problemsolvingwithpython.com/01-Orientation/01.05-Installing-Anaconda-on-Linux/}
\begin{quote}
\begin{description}
\item[{In shell do following:}] \leavevmode
\$ curl \sphinxurl{https://repo.anaconda.com/archive/Anaconda3-2020.02-Linux-x86\_64.sh} \sphinxhyphen{}o \textasciitilde{}/Downloads/Anaconda3\sphinxhyphen{}2020.02\sphinxhyphen{}Linux\sphinxhyphen{}x86\_64.sh
\$ sudo chmod 777 /opt
\$ bash \textasciitilde{}/Downloads/Anaconda3\sphinxhyphen{}2020.02\sphinxhyphen{}Linux\sphinxhyphen{}x86\_64.sh

\begin{sphinxVerbatim}[commandchars=\\\{\}]
\PYG{g+gp}{\PYGZgt{}\PYGZgt{}\PYGZgt{} }\PYG{p}{[}\PYG{n}{Enter}\PYG{p}{]}
\PYG{g+go}{[s]}
\PYG{g+gp}{\PYGZgt{}\PYGZgt{}\PYGZgt{} }\PYG{n}{yes}
\PYG{g+gp}{\PYGZgt{}\PYGZgt{}\PYGZgt{} }\PYG{o}{/}\PYG{n}{opt}\PYG{o}{/}\PYG{n}{anaconda3}
\PYG{g+gp}{\PYGZgt{}\PYGZgt{}\PYGZgt{} }\PYG{n}{yes}
\end{sphinxVerbatim}

\$ sudo chmod 755 /opt
\$ sudo reboot

\end{description}
\end{quote}
\begin{enumerate}
\sphinxsetlistlabels{\arabic}{enumi}{enumii}{}{.}%
\setcounter{enumi}{4}
\item {} 
Anaconda: create sentinel\sphinxhyphen{}hindcast\sphinxhyphen{}37 environment
\begin{quote}
\begin{description}
\item[{In shell do following:}] \leavevmode
\$ conda config \textendash{}add channels conda\sphinxhyphen{}forge
\$ conda create \textendash{}name sentinel\sphinxhyphen{}hindcast\sphinxhyphen{}37 python=3.7 gdal cartopy netcdf4 cython pkgconfig statsmodels matplotlib haversine
\begin{quote}

\textgreater{} y
\end{quote}

\end{description}
\end{quote}

\end{enumerate}

6.) SNAP: \sphinxurl{http://step.esa.int/main/download/}
\begin{quote}

Uninstall all old versions of SNAP and remove associated data
\begin{description}
\item[{In shell do following:}] \leavevmode
\$ curl \sphinxurl{http://step.esa.int/downloads/7.0/installers/esa-snap\_all\_unix\_7\_0.sh} \sphinxhyphen{}o \textasciitilde{}/Downloads/esa\sphinxhyphen{}snap\_all\_unix\_7\_0.sh
\$ sudo chmod 777 /opt
\$ bash \textasciitilde{}/Downloads/esa\sphinxhyphen{}snap\_all\_unix\_7\_0.sh
\begin{quote}

{[}o, Enter{]}
{[}1, Enter{]}
{[}Enter{]}
{[}Enter{]}
{[}n, Enter{]}
{[}n, Enter{]}
{[}n, Enter{]}
\end{quote}

\$ sudo chmod 755 /opt
\$ /opt/snap/bin/snap \textendash{}nosplash \textendash{}nogui \textendash{}modules \textendash{}update\sphinxhyphen{}all
\$ /opt/snap/bin/snap \textendash{}nosplash \textendash{}nogui \textendash{}modules \textendash{}install org.esa.snap.idepix.core org.esa.snap.idepix.probav org.esa.snap.idepix.modis org.esa.snap.idepix.spotvgt org.esa.snap.idepix.landsat8 org.esa.snap.idepix.viirs org.esa.snap.idepix.olci org.esa.snap.idepix.seawifs org.esa.snap.idepix.meris org.esa.snap.idepix.s2msi
\$ echo “\#SNAP configuration ‘s3tbx’” \textgreater{}\textgreater{} \textasciitilde{}/.snap/etc/s3tbx.properties
\$ echo “\#Fri Mar 27 12:55:00 CET 2020” \textgreater{}\textgreater{} \textasciitilde{}/.snap/etc/s3tbx.properties
\$ echo “s3tbx.reader.olci.pixelGeoCoding=true” \textgreater{}\textgreater{} \textasciitilde{}/.snap/etc/s3tbx.properties
\$ echo “s3tbx.reader.meris.pixelGeoCoding=true” \textgreater{}\textgreater{} \textasciitilde{}/.snap/etc/s3tbx.properties
\$ echo “s3tbx.reader.slstrl1b.pixelGeoCodings=true” \textgreater{}\textgreater{} \textasciitilde{}/.snap/etc/s3tbx.properties

\end{description}

Note: there are many strange error messages, but it seems to work in the end when updating and installing plugins
\begin{description}
\item[{To remove warning “WARNING: org.esa.snap.dataio.netcdf.util.MetadataUtils: Missing configuration property ‘snap.dataio.netcdf.metadataElementLimit’. Using default (100).”:}] \leavevmode
\$ echo “” \textgreater{}\textgreater{} /opt/snap/etc/snap.properties
\$ echo “\# NetCDF options” \textgreater{}\textgreater{} /opt/snap/etc/snap.properties
\$ echo “snap.dataio.netcdf.metadataElementLimit=10000” \textgreater{}\textgreater{} /opt/snap/etc/snap.properties

\item[{To remove warning “SEVERE: org.esa.s2tbx.dataio.gdal.activator.GDALDistributionInstaller: The environment variable LD\_LIBRARY\_PATH is not set. It must contain the current folder ‘.’.”}] \leavevmode
\$ sudo su \sphinxhyphen{}c ‘echo “LD\_LIBRARY\_PATH=.” \textgreater{}\textgreater{} /etc/environment’

\end{description}
\end{quote}

7.) Python \sphinxhyphen{} jpy: \sphinxurl{https://github.com/bcdev/jpy/blob/master/README.md}
\begin{quote}
\begin{description}
\item[{In shell do following:}] \leavevmode
\$ cd /opt/anaconda3/envs/sentinel\sphinxhyphen{}hindcast\sphinxhyphen{}37/lib/python3.7/site\sphinxhyphen{}packages
\$ git clone \sphinxurl{https://github.com/bcdev/jpy.git}
\$ cd jpy
\$ conda activate sentinel\sphinxhyphen{}hindcast\sphinxhyphen{}37
\$ python get\sphinxhyphen{}pip.py
\$ python setup.py build maven bdist\_wheel

\end{description}
\end{quote}

8.) Python \sphinxhyphen{} snappy: \sphinxurl{https://github.com/senbox-org/snap-engine/blob/master/snap-python/src/main/resources/README.md}
\begin{quote}
\begin{description}
\item[{In shell do following:}] \leavevmode
\$ sudo ln \sphinxhyphen{}s ../../lib64/libnsl.so.2 /usr/lib64/libnsl.so
\$ sudo ln \sphinxhyphen{}s ../../lib64/libnsl.so.2.0.0 /usr/lib64/libnsl.so.1
\$ mkdir \sphinxhyphen{}p \textasciitilde{}/.snap/snap\sphinxhyphen{}python/snappy
\$ cp /opt/anaconda3/envs/sentinel\sphinxhyphen{}hindcast\sphinxhyphen{}37/lib/python3.7/site\sphinxhyphen{}packages/jpy/dist/{\color{red}\bfseries{}*}.whl \textasciitilde{}/.snap/snap\sphinxhyphen{}python/snappy
\$ bash /opt/snap/bin/snappy\sphinxhyphen{}conf /opt/anaconda3/envs/sentinel\sphinxhyphen{}hindcast\sphinxhyphen{}37/bin/python \textasciitilde{}/.snap/snap\sphinxhyphen{}python
\$ conda activate sentinel\sphinxhyphen{}hindcast\sphinxhyphen{}37
\$ python \textasciitilde{}/.snap/snap\sphinxhyphen{}python/snappy/setup.py install \textendash{}user
\$ cp \sphinxhyphen{}avr \textasciitilde{}/.snap/snap\sphinxhyphen{}python/build/lib/snappy /opt/anaconda3/envs/sentinel\sphinxhyphen{}hindcast\sphinxhyphen{}37/lib/python3.7/site\sphinxhyphen{}packages/snappy
\$ cp \sphinxhyphen{}avr \textasciitilde{}/.snap/snap\sphinxhyphen{}python/snappy/tests /opt/anaconda3/envs/sentinel\sphinxhyphen{}hindcast\sphinxhyphen{}37/lib/python3.7/site\sphinxhyphen{}packages/snappy/tests
\$ cd /opt/anaconda3/envs/sentinel\sphinxhyphen{}hindcast\sphinxhyphen{}37/lib/python3.7/site\sphinxhyphen{}packages/snappy/tests
\$ curl \sphinxurl{https://raw.githubusercontent.com/bcdev/eo-child-gen/master/child-gen-N1/src/test/resources/com/bc/childgen/MER\_RR\_\_1P.N1} \sphinxhyphen{}o MER\_RR\_\_1P.N1
\$ python test\_snappy\_mem.py
\$ python test\_snappy\_perf.py
\$ python test\_snappy\_product.py

\end{description}
\end{quote}

9.) Python \sphinxhyphen{} polymer: \sphinxurl{https://forum.hygeos.com/viewforum.php?f=5}
\begin{quote}
\begin{description}
\item[{From a computer in the eawag network, copy the polymer zip file to the linux server:}] \leavevmode
\textgreater{} scp \sphinxhyphen{}i .sshcloudferro.key \textbackslash{}eawagAbteilungs\sphinxhyphen{}ProjekteSurfsurf\sphinxhyphen{}DDRSSoftwarePolymerpolymer\sphinxhyphen{}v4.13.tar.gz \sphinxhref{mailto:eouser@45.130.29.115}{eouser@45.130.29.115}:/home/eouser/Downloads

\item[{In shell do following:}] \leavevmode
\$ sudo chmod 777 /opt
\$ tar \sphinxhyphen{}xvzf \textasciitilde{}/Downloads/polymer\sphinxhyphen{}v4.13.tar.gz \textendash{}directory /opt
\$ sudo chmod 755 /opt
\$ cd /opt/polymer\sphinxhyphen{}v4.13
\$ conda activate sentinel\sphinxhyphen{}hindcast\sphinxhyphen{}37
\$ bash install\sphinxhyphen{}anaconda\sphinxhyphen{}deps.sh
\begin{quote}

{[}y{]}
{[}y{]}
\end{quote}

\$ sudo yum install wget
\$ make all
\$ cp \sphinxhyphen{}avr /opt/polymer\sphinxhyphen{}v4.13/polymer /opt/anaconda3/envs/sentinel\sphinxhyphen{}hindcast\sphinxhyphen{}37/lib/python3.7/site\sphinxhyphen{}packages/polymer
\$ cp \sphinxhyphen{}avr /opt/polymer\sphinxhyphen{}v4.13/auxdata /opt/anaconda3/envs/sentinel\sphinxhyphen{}hindcast\sphinxhyphen{}37/lib/python3.7/site\sphinxhyphen{}packages/auxdata

\end{description}
\end{quote}

10.) sentinel\sphinxhyphen{}hindcast: \sphinxurl{https://renkulab.io/gitlab/odermatt/sentinel-hindcast}
\begin{quote}
\begin{description}
\item[{In shell do following:}] \leavevmode
\$ cd /prj
\$ sudo chmod 777 /prj
\$ mkdir /prj/DIAS
\$ git clone \sphinxurl{https://renkulab.io/gitlab/odermatt/sentinel-hindcast.git}
\$ sudo chmod 755 /prj
\$ cd sentinel\sphinxhyphen{}hindcast
\$ git checkout \textless{}branchname\textgreater{} (if not master)

\end{description}
\end{quote}

11.) CDS API: \sphinxurl{https://cds.climate.copernicus.eu/api-how-to}
\begin{quote}

Have a Copernicus Climate account ready, otherwise create one: \sphinxurl{https://cds.climate.copernicus.eu/}
\begin{description}
\item[{In shell do following:}] \leavevmode
\$ echo “url: \sphinxurl{https://cds.climate.copernicus.eu/api/v2}” \textgreater{} \textasciitilde{}/.cdsapirc
\$ echo key: {[}uid{]}:{[}api\sphinxhyphen{}key{]} \textgreater{}\textgreater{} \textasciitilde{}/.cdsapirc (Note: replace {[}uid{]} and {[}api\sphinxhyphen{}key{]} by your actual credentials, see \sphinxurl{https://cds.climate.copernicus.eu/api-how-to} )
\$ chmod 600 \textasciitilde{}/.cdsapirc

\end{description}
\end{quote}

12.) Cronjob for datalakes: \sphinxurl{https://linux4one.com/how-to-set-up-cron-job-on-centos-8/}
\begin{quote}
\begin{description}
\item[{In shell do following:}] \leavevmode
\$ mkdir \sphinxhyphen{}p /prj/datalakes/log
\$ curl \sphinxurl{https://renkulab.io/gitlab/odermatt/sentinel-hindcast/raw/snap7compatibility/parameters/datalakes\_sui\_S3.ini?inline=false} \sphinxhyphen{}o /prj/datalakes/datalakes\_sui\_S3.ini
\$ chmod 755 /prj/sentinel\sphinxhyphen{}hindcast/scripts/datalakes.sh
\$ crontab \sphinxhyphen{}l | \{ cat; echo “0 20 * * * nohup /prj/sentinel\sphinxhyphen{}hindcast/scripts/datalakes.sh \&”; \} | crontab \sphinxhyphen{}

\end{description}
\end{quote}

13.) (not done yet) NASA Earthdata API: \sphinxurl{https://wiki.earthdata.nasa.gov/display/EL/How+To+Access+Data+With+cURL+And+Wget}
\begin{quote}

Have a NASA Earthdata account ready, otherwise create one: \sphinxurl{https://urs.earthdata.nasa.gov/}
\begin{description}
\item[{In shell do following:}] \leavevmode
\$ touch \textasciitilde{}/.netrc
\$ echo “machine urs.earthdata.nasa.gov login \textless{}earthdata user\textgreater{} password \textless{}earthdata password\textgreater{}” \textgreater{} \textasciitilde{}/.netrc
\$ chmod 0600 \textasciitilde{}/.netrc
\$ touch \textasciitilde{}/.urs\_cookies

\end{description}
\end{quote}


\subsection{Windows 10}
\label{\detokenize{windows10_install:windows-10}}\label{\detokenize{windows10_install:windows10install}}\label{\detokenize{windows10_install::doc}}
1.) baramundi Kiosk: \sphinxurl{http://soft-prd:10080/Softwarekiosk/default.htm}
\begin{quote}

Get following programs from baramundi Kiosk:
\sphinxhyphen{} Git
\sphinxhyphen{} OpenJdk 1.8
\end{quote}

2.) JAVA\_HOME:
\begin{quote}

Set JAVA\_HOME to “C:path\_to\_jdk" (e.g. C:Program FilesAdoptOpenJDKjdk\sphinxhyphen{}8.0.242.08\sphinxhyphen{}hotspot)

Add “\%JAVA\_HOME\%bin” to PATH

Set JDK\_HOME to “\%JAVA\_HOME\%”
\end{quote}

3.) Maven: \sphinxurl{http://maven.apache.org/download.cgi}
\begin{quote}

Download Maven 3.6.3

Unzip installation folder and move it to the desired directory (preferred C:Program Files (x86)apache\sphinxhyphen{}maven\sphinxhyphen{}3.6.3)

Set MAVEN\_HOME to “C:path\_to\_maven\_folder" (e.g. C:Program Files (x86)apache\sphinxhyphen{}maven\sphinxhyphen{}3.6.3)

Add “\%MAVEN\_HOME\%bin” to PATH

Check with: mvn \textendash{}version
\end{quote}

4.) Visual C++ Build Tools 2015: \sphinxurl{https://go.microsoft.com/fwlink/?LinkId=691126}

5.) Anaconda: \sphinxurl{https://www.anaconda.com/distribution/}
\begin{quote}

Download Anaconda3

Install Anaconda3 (prefered to C:Anaconda3)

Set CONDA\_HOME to “C:path\_to\_anaconda\_installation" (e.g. C:Anaconda3)

Add “\%CONDA\_HOME\%bin” to PATH  (could need to be “\%CONDA\_HOME\%condabin”)
\end{quote}
\begin{enumerate}
\sphinxsetlistlabels{\arabic}{enumi}{enumii}{}{.}%
\setcounter{enumi}{5}
\item {} 
Anaconda: sentinel\sphinxhyphen{}hindcast\sphinxhyphen{}37 environment
\begin{quote}
\begin{description}
\item[{Add conda\sphinxhyphen{}forge channel to environment:}] \leavevmode
\textgreater{} conda config \textendash{}add channels conda\sphinxhyphen{}forge

\item[{Create a new environment named “sentinel\sphinxhyphen{}hindcast” using Anaconda}] \leavevmode
\textgreater{} conda create \textendash{}name sentinel\sphinxhyphen{}hindcast\sphinxhyphen{}37 python=3.7 gdal cartopy netcdf4 cython pkgconfig statsmodels matplotlib haversine

\end{description}

Set CONDA\_ENV\_HOME to “\%CONDA\_HOME\%envssentinel\sphinxhyphen{}hindcast\sphinxhyphen{}37”
\end{quote}

\end{enumerate}

6.) Python \sphinxhyphen{} jpy: \sphinxurl{https://github.com/bcdev/jpy/blob/master/README.md}
\begin{quote}
\begin{description}
\item[{Start a command prompt and do following:}] \leavevmode
\textgreater{} cd “\%CONDA\_ENV\_HOME\%Libsite\sphinxhyphen{}packages”
\textgreater{} git clone \sphinxurl{https://github.com/bcdev/jpy.git}
\textgreater{} cd “jpy”
\textgreater{} conda activate sentinel\sphinxhyphen{}hindcast\sphinxhyphen{}37
\textgreater{} python get\sphinxhyphen{}pip.py
\textgreater{} python setup.py build maven bdist\_wheel

\end{description}
\end{quote}

7.) SNAP: \sphinxurl{http://step.esa.int/main/download/}
\begin{quote}
\begin{description}
\item[{Uninstall all old versions of SNAP:}] \leavevmode\begin{itemize}
\item {} 
Uninstall via “Control Panel \sphinxhyphen{}\textgreater{} Programs and Features”

\item {} 
Choose to delete all user data

\item {} 
Check that the SNAP installation folder has been removed completely by uninstalling. Otherwise delete it manually.

\item {} 
Delete snappy folder from all your python environments: \%PYTHON\_HOME\%Libsite\sphinxhyphen{}packages

\item {} 
Delete .snap folder from all user accounts: \%USERPROFILE\%.snap

\item {} 
Delete SNAP Folder from all user accoutns: \%USERPROFILE\%AppDataRoamingSNAP

\end{itemize}

\end{description}

Download SNAP

Install SNAP

Do not configure SNAP for use with Python yet.

Run SNAP and install available updates.

Configure: Tools \sphinxhyphen{}\textgreater{} Options \sphinxhyphen{}\textgreater{} S3TBX \sphinxhyphen{}\textgreater{} Check: Read Sentinel\sphinxhyphen{}3 OLCI products with per pixel geo\sphinxhyphen{}coding instead of using tie\sphinxhyphen{}points

Configure: Tools \sphinxhyphen{}\textgreater{} Plugins \sphinxhyphen{}\textgreater{} Available Plugins \sphinxhyphen{}\textgreater{} Install all IDEPIX Plugins

Set SNAP\_HOME to “C:path\_to\_snap\_isntallation" (e.g. C:Snap7)

Close SNAP
\end{quote}

8.) Python \sphinxhyphen{} snappy: \sphinxurl{https://github.com/senbox-org/snap-engine/blob/master/snap-python/src/main/resources/README.md}
\begin{quote}
\begin{description}
\item[{Start a command prompt and do following:}] \leavevmode
\textgreater{} cd “\%SNAP\_HOME\%bin”
\textgreater{} xcopy “\%CONDA\_ENV\_HOME\%Libsite\sphinxhyphen{}packagesjpydist*.whl” “\%USERPROFILE\%.snapsnap\sphinxhyphen{}pythonsnappy"
\textgreater{} snappy\sphinxhyphen{}conf “\%CONDA\_ENV\_HOME\%python.exe” “\%USERPROFILE\%.snapsnap\sphinxhyphen{}python”
\textgreater{} (I had to end the process after about 30 seconds using “Ctrl+C” at this point)
\textgreater{} cd “\%USERPROFILE\%.snapsnap\sphinxhyphen{}pythonsnappy”
\textgreater{} conda activate sentinel\sphinxhyphen{}hindcast\sphinxhyphen{}37
\textgreater{} python setup.py install
\textgreater{} xcopy “\%USERPROFILE\%.snapsnap\sphinxhyphen{}pythonsnappytests” “\%CONDA\_ENV\_HOME\%Libsite\sphinxhyphen{}packagessnappytests"
\textgreater{} cd “\%CONDA\_ENV\_HOME\%Libsite\sphinxhyphen{}packagessnappytests”
\textgreater{} curl \textendash{}url “\sphinxurl{https://raw.githubusercontent.com/bcdev/eo-child-gen/master/child-gen-N1/src/test/resources/com/bc/childgen/MER\_RR\_\_1P.N1}” \textendash{}output “\%CONDA\_ENV\_HOME\%Libsite\sphinxhyphen{}packagessnappytestsMER\_RR\_\_1P.N1”
\textgreater{} python test\_snappy\_mem.py
\textgreater{} python test\_snappy\_perf.py
\textgreater{} python test\_snappy\_product.py

\end{description}
\end{quote}

9.) Python \sphinxhyphen{} polymer: \sphinxurl{https://forum.hygeos.com/viewforum.php?f=5}
\begin{quote}
\begin{description}
\item[{Start a command prompt and do following:}] \leavevmode
\textgreater{} cd “\%USERPROFILE\%AppDataLocalTemp”
\textgreater{} xcopy “Q:AbteilungsprojekteSurfsurf\sphinxhyphen{}DDRSSoftwarePolymerpolymer\sphinxhyphen{}v4.13.zip” “\%USERPROFILE\%AppDataLocalTemp”
\textgreater{} jar xf “polymer\sphinxhyphen{}v4.13.zip”
\textgreater{} cd “polymer\sphinxhyphen{}v4.13”
\textgreater{} conda activate sentinel\sphinxhyphen{}hindcast\sphinxhyphen{}37
\textgreater{} conda install pyhdf pyepr glymur pygrib cdsapi xarray bioconda::ecmwfapi
\textgreater{} python setup.py build\_ext \textendash{}inplace
\textgreater{} xcopy “\%USERPROFILE\%AppDataLocalTemppolymer\sphinxhyphen{}v4.13polymer” “\%CONDA\_ENV\_HOME\%Libsite\sphinxhyphen{}packagespolymer"
\textgreater{} xcopy “\%USERPROFILE\%AppDataLocalTemppolymer\sphinxhyphen{}v4.13auxdata” “\%CONDA\_ENV\_HOME\%Libsite\sphinxhyphen{}packagesauxdata"

\end{description}
\end{quote}

10.) sentinel\sphinxhyphen{}hindcast: \sphinxurl{https://renkulab.io/gitlab/odermatt/sentinel-hindcast}
\begin{quote}
\begin{description}
\item[{Start a command prompt and do following:}] \leavevmode
\textgreater{} cd “C:Projects”
\textgreater{} mkdir “DIAS”
\textgreater{} mkdir “datalakes”
\textgreater{} git clone https://renkulab.io/gitlab/odermatt/sentinel\sphinxhyphen{}hindcast.git

\end{description}
\end{quote}

11.) CDS API: \sphinxurl{https://cds.climate.copernicus.eu/api-how-to}
\begin{quote}
\begin{description}
\item[{Start a command prompt and do following:}] \leavevmode
\textgreater{} echo url: \sphinxurl{https://cds.climate.copernicus.eu/api/v2} \textgreater{} \%USERPROFILE\%.cdsapirc
\textgreater{} echo key: \textless{}uid\textgreater{}:\textless{}api\sphinxhyphen{}key\textgreater{} \textgreater{}\textgreater{} \%USERPROFILE\%.cdsapirc

\end{description}
\end{quote}

12.) PyCharm CE: \sphinxurl{https://www.jetbrains.com/de-de/pycharm/download/\#section=windows}
\begin{quote}

Download PyCharm CE from \sphinxurl{https://www.jetbrains.com/de-de/pycharm/download/download-thanks.html?platform=windows\&code=PCC}

Install PyCharm CE with default settings

Launch PyCharm CE

Open \sphinxhyphen{}\textgreater{} C:Projectssentinel\sphinxhyphen{}hindcast
\begin{description}
\item[{Add a Project Interpreter}] \leavevmode\begin{itemize}
\item {} 
File \sphinxhyphen{}\textgreater{} Settings \sphinxhyphen{}\textgreater{} Project: sentinel\sphinxhyphen{}hindcast \sphinxhyphen{}\textgreater{} Gearwheel in the upper right \sphinxhyphen{}\textgreater{} Show All…

\item {} 
Add (+) \sphinxhyphen{}\textgreater{} Conda Environment \sphinxhyphen{}\textgreater{} Existing environment \sphinxhyphen{}\textgreater{} Interpreter: C:Anaconda3envssentinel\sphinxhyphen{}hindcast\sphinxhyphen{}37python.exe \sphinxhyphen{}\textgreater{} OK \sphinxhyphen{}\textgreater{} OK \sphinxhyphen{}\textgreater{} OK

\item {} 
Give it some time to index files (watch processes in the bottom line to finish)

\end{itemize}

\item[{Define a running configuration:}] \leavevmode\begin{itemize}
\item {} 
In the top right “Add Configuration…”

\item {} 
In the top left Add (+) \sphinxhyphen{}\textgreater{} Python

\item {} 
Name: sentinel\sphinxhyphen{}hindcast\sphinxhyphen{}37

\item {} 
Script path: C:Projectssentinel\sphinxhyphen{}hindcastsentinel\sphinxhyphen{}hindcast.py

\item {} 
Python interpreter: Python 3.7 (sentinel\sphinxhyphen{}hindcast\sphinxhyphen{}37)

\item {} 
OK

\end{itemize}

\end{description}
\end{quote}

You are now set up and ready to start coding as well as running sentinel\sphinxhyphen{}hindcast


\subsection{Environment File}
\label{\detokenize{environment_config:environment-file}}\label{\detokenize{environment_config:environments}}\label{\detokenize{environment_config::doc}}

\subsection{Parameter File}
\label{\detokenize{parameters_config:parameter-file}}\label{\detokenize{parameters_config:parameters}}\label{\detokenize{parameters_config::doc}}

\subsection{c2rcc}
\label{\detokenize{c2rcc:c2rcc}}\label{\detokenize{c2rcc::doc}}

\subsection{fluo}
\label{\detokenize{fluo:fluo}}\label{\detokenize{fluo::doc}}

\subsection{Idepix}
\label{\detokenize{idepix:idepix}}\label{\detokenize{idepix::doc}}

\subsection{Mosaic}
\label{\detokenize{mosaic:mosaic}}\label{\detokenize{mosaic::doc}}

\subsection{MPH}
\label{\detokenize{mph:mph}}\label{\detokenize{mph::doc}}

\subsection{Polymer}
\label{\detokenize{polymer:polymer}}\label{\detokenize{polymer::doc}}

\subsection{Sen2cor}
\label{\detokenize{sen2cor:sen2cor}}\label{\detokenize{sen2cor::doc}}

\subsection{sentinel\sphinxhyphen{}hindcast}
\label{\detokenize{modules:sentinel-hindcast}}\label{\detokenize{modules::doc}}

\subsubsection{sentinelhindcast package}
\label{\detokenize{sentinelhindcast:sentinelhindcast-package}}\label{\detokenize{sentinelhindcast::doc}}

\paragraph{Submodules}
\label{\detokenize{sentinelhindcast:submodules}}

\paragraph{sentinelhindcast.core module}
\label{\detokenize{sentinelhindcast:sentinelhindcast-core-module}}

\paragraph{sentinelhindcast.auxil module}
\label{\detokenize{sentinelhindcast:sentinelhindcast-auxil-module}}

\paragraph{sentinelhindcast.helper module}
\label{\detokenize{sentinelhindcast:sentinelhindcast-helper-module}}

\paragraph{Module contents}
\label{\detokenize{sentinelhindcast:module-contents}}

\chapter{Indices and tables}
\label{\detokenize{index:indices-and-tables}}\begin{itemize}
\item {} 
\DUrole{xref,std,std-ref}{genindex}

\item {} 
\DUrole{xref,std,std-ref}{modindex}

\item {} 
\DUrole{xref,std,std-ref}{search}

\end{itemize}



\renewcommand{\indexname}{Index}
\printindex
\end{document}